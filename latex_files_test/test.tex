\documentclass[11pt]{amsbook}
\bibliographystyle{amsalpha}
\begin{document}

\section{Motivation: assessing effectivity of urban methods}
\label{sec-Motivation}

\subsection*{Case studies}
\begin{itemize}
  \item Cape Town. 
\end{itemize}

\subsection*{Main question}

\noindent Hypothesis: 
\begin{itemize}
  \item nature based solutions is necessary. 
\end{itemize}

\noindent Is this executable? 


\begin{itemize}
  \item How do we assess how trees contribute to human well being? 
\end{itemize}
\noindent One needs a way to assess the value attained. We put this into three aspects. 
\begin{enumerate}
  \item Biophysical and ecological. 
  \item  Economic and financial. 
  \item Social and cultural,  see \ref{sec-metrics-well-being}. 
\end{enumerate}

\noindent The latter two are often less studied. 
\begin{tabular}{c|c|c|c|c}
  Local climate & \begin{tabular}{
    c|c
  }
  green space cover \\ 
    temperature 
  \end{tabular}
\end{tabular}


\section{Metrics for well-being}
\label{sec-metrics-well-being}
\noindent What is well-being? 
\noindent There are several metrics used by current researchers. 

\begin{itemize}
  \item 
\end{itemize}


\section{Recommendations}


\noindent This


\end{document}


